% Typeset using lualatex

\documentclass[a4paper,10pt]{article}
\usepackage[utf8]{luainputenc}
\usepackage[pdfborder={0 0 0}]{hyperref}

\usepackage[czech]{babel}

\usepackage{fouriernc}
\usepackage{tgpagella}
\usepackage{xcolor}
\usepackage[activate=true]{microtype}
\usepackage{amssymb}
\usepackage{fontspec}
\usepackage{titling}
\usepackage{titlesec}

\usepackage{calc}
\usepackage{multicol}
\usepackage{array}
\usepackage{enumitem}
\usepackage{tikz}
\usepackage{siunitx}
\usepackage{colortbl}
\usepackage[a4paper,margin=2cm,top=1cm]{geometry}

\setmainfont[Numbers=OldStyle]{TeX Gyre Pagella}

\definecolor{plhome}{HTML}{CCEEFF}
\definecolor{silver}{HTML}{DDDDDD}

\sisetup{
    output-decimal-marker={,}
}
\pagestyle{empty}

\parskip=1ex
\parindent=0pt

\setlist[enumerate,1]{start=0}

\newcommand\plpage{
    \newpage
    \begin{tikzpicture}[remember picture,overlay]
    \node[opacity=0.1,below left,xshift=1cm,yshift=1cm] at (current page.north east) {\includegraphics[width=12cm]{../../images/pylady-pink}};
    \end{tikzpicture}
}

\newcommand\answerspace{\\\rule[0cm]{0pt}{1cm}}

\newcommand\True{\texttt{True}}
\newcommand\False{\texttt{False}}

\newcommand\startsection[1]{
     \vspace{0.2ex}
    \hrule
    {\fontspec{Oxygen} \tiny
     \vspace{-1ex}
     \emph{#1}
     \vspace{-1.5em}
    }
}

\newfontfamily\headingfont[]{Bree Serif}
\titleformat*{\section}{\LARGE\headingfont}

\begin{document}

\plpage

\section*{Domácí úkoly 2}

\begin{enumerate}
\item Jakou používáš verzi Pythonu?

\end{enumerate}

\startsection{Následující úkoly procvičí orientaci v chybových hláškách. Doporučuji je udělat.}

\begin{enumerate}[resume]

\item Jak se jmenuje druh chyby, která nastane, když...
    \begin{itemize}
        \item Dáš uvozovky jen na jednu stranu řetězce? — \texttt{SyntaxError}
        \item Zkusíš odečíst číslo od řetězce? —
        \item Dělíš nulou? —
        \item Použiješ proměnnou, která neexistuje? —
        \item Stiskneš Ctrl+C, když se program ptá na vstup (pomocí \texttt{input})? —
        \item Odsadíš příkaz bez předchozího \texttt{if:}? —
        \item Po \texttt{if:} odsadíš jeden příkaz o čtyři mezery, a druhý jen o dvě? —
        \item Neuzavřeš závorku? –
        \item Zkusíš použít vykřičník (\texttt{!}) jako operátor? —
        \item Napíšeš v příkazu \texttt{print(1, 2, 3)} čárku navíc? —
    \end{itemize}

\item \emph{\small Podívej se na odpovědi na předchozí otázku, ale Python zkus použít jen na ověření.}
    \\Jaká chyba nastane, když zkusíš použít proměnnou předtím, než do ní něco přiřadíš?

\item \emph{\small Podívej se na odpovědi na otázku 1, ale Python použij jen na ověření.}
    \\Jaká chyba nastane, když zkusíš podělit řetězec řetězcem?

\item Lomítko (\texttt{/}) je operátor, ale nedá se použít na řetězce. Vykřičník (\texttt{!}) v Pythonu není operátor.
    \\Jak se liší „jejich“ chyby? \emph{\small Rozdíly jsou aspoň dva.}

\item Je zavináč (\texttt{@}) v Pythonu operátor?  % odpověď na tohle se v Pythonu 3.5 změní. To bude sranda!

\end{enumerate}

\startsection{Tenhle úkol by měl přivést k tomu, jak zjišťovat co v Pythonu jde a co ne.
Druhá část je trochu na zamyšlení.}

\begin{enumerate}[resume]
\item Ne všechno se dá použít jako jméno proměnné.
    \\Fungují pro proměnné následující jména? Pokud ne, proč asi?
    {
        \newcommand\varname[2][]{\varnameend[#1]{#2}\\[0.25cm]}
        \newcommand\varnameend[2][]{\texttt{#2} #1}
        \begin{multicols}{4}
        \varname{x}
        \varname{tlacitko4}
        \varname{34}
        \varname{3e4}
        \varname{krůta}
        \varname{\$i}
        \varname{druha-odmocnina}
        \varname{next.file}
        \varname{kratsiStrana}
        \varname{POCET\_BODU}
        \varname[(podtržítko)]{\_}
        \varname[(pí)]{\textrm{π}}
        \varname{True}
        \varname{\_cache}
        \varname{\_\_name\_\_}
        \varnameend{while}
        \end{multicols}
    }

\end{enumerate}

\startsection{Následující úkoly jsou na procvičení toho, co jsme dělaly na srazu.
Nemáš-li čas, zatím je přeskoč.}

\begin{enumerate}[resume]

\item \emph{\small Zkus se nedívat na programy ze srazu.}
    \\Napiš program, který spočítá povrch a obsah krychle o straně \SI{2852}{cm}.
    \\Abys nemusela tolik hledat v učebnici (vlastně Wikipedii): povrch $S=6a^2$, obsah $V=a^3$
    \\\emph{\small Řešení, pro kontrolu: $S=\SI{48803424}{cm^2}$, $V=\SI{23197894208}{cm^3}$}

\item Změň program tak, aby stranu/poloměr mohl uživatel zadat.
    \\\emph{\small Tady už se na materiály ke srazu klidně podívej.}

\end{enumerate}

\startsection{Další dva úkoly doplňují program ze srazu, a jsou úplně nepovinné.}

\begin{enumerate}[resume]

\item Zkus pustit tento program několikrát za sebou. Co dělá?
    \\\verb+from random import randrange+
    \\\verb+cislo = randrange(3)+
    \\\verb+print(cislo)+
    \\\emph{\small Jak to funguje, to se dozvíme příště; zatím to ber jako kouzelné zaříkadlo.}

\item Zkombinuj program z předchozího úkolu s programem kámen–nůžky–papír, a nastav \verb+tah_pocitace+ na:
    \begin{enumerate}
        \item \verb+'kámen'+, \emph{pokud} je \verb+cislo+ 0
        \item \verb+'nůžky'+, pokud je \verb+cislo+ 1
        \item jinak na \verb+'papír'+
    \end{enumerate}

\end{enumerate}

\plpage

\startsection{Tyhle úkoly kombinují opakování a nové informace.
Doporučuji si je projít.}

\begin{enumerate}[resume]

\item Které Pythoní \emph{operátory} dokážeš z hlavy vyjmenovat?

\item Zkusila jsi porovnávat řetězce?
    \\Doplň tuhle tabulku tužkou; pak ověř odpovědi pomocí Pythonu:

    {
        \newcommand\rowend{\rule{0pt}{0.5cm}\\ \hline}
        \begin{tabular}{c|c|c}
        \arrayrulecolor{silver}
        \verb+a+ &  & \verb+b+ \\
        \arrayrulecolor{black}\hline\arrayrulecolor{silver}
        \verb+2+ & < & \verb+1+ \rowend
        \verb+1+ & > & \verb+2+ \rowend
        \verb+'abc'+ & == & \verb+'abc'+ \rowend
        \verb+'aaa'+ &    & \verb+'abc'+ \rowend
        \verb+'abc'+ &    & \verb+'Abc'+ \rowend
        \verb+'abC'+ &    & \verb+'abc'+ \rowend
        \verb+'abc'+ &    & \verb+'abcde'+ \rowend
        \verb+'abc'+ &    & \verb+'ábč'+ \rowend
        \verb+'abc'+ &    & \verb+10+ \rowend
        \end{tabular}
    }

\item Jaká je hodnota proměnné po provedení příkazu:
    \\\verb+promenna = 1 < 2+ \hspace{0.2cm}?
    \\\verb+promenna = 2 < 2+ \hspace{0.2cm}?
    \\\verb+promenna = 1 < 2 < 3+ \hspace{0.2cm}?
    \\\verb+promenna = 1 < 3 < 2+ \hspace{0.2cm}?
    \\\verb+promenna = 1 < 3 < 3+ \hspace{0.2cm}?
    \\\verb+promenna = 'abc' < 'ABC' < 'def' < 'zajíc'+ \hspace{0.2cm}?
    \\Tam kde je hodnota proměnné \False, dají se některé z porovnávaných hodnot vyměnit, aby byla \True?

\end{enumerate}

\startsection{Další dva úkoly ti umožní kreativně využít co ses naučila.}

\begin{enumerate}[resume]

\item Na srazu jsme měli program, který píše různé nesmysly podle uživatelem zadaného věku.
    \\Zkus napsat program, který píše hlášky podle zadané
    rychlosti chůze, váhy ulovené ryby, počtu tykadel, teploty vody
    nebo třeba vzdálenosti od rovníku.
    \\Program pošli na soukromý e-mail organizátora (ne do diskusní skupiny).
    Pošli ho jako přílohu, nekopíruj ho do textu e-mailu.

\item Napiš program, který po zadání správného hesla vypíše nějakou tajnou informaci.
    \\\emph{\small Vhodné tajemství je třeba: V pátek jsem viděla černého havrana.}

\end{enumerate}

\startsection{Poslední úkoly jsou pokročilejší.
Jestli nemáš čas, radši udělej pořádně ty ostatní.}

\begin{enumerate}[resume]

\item Projdi si \url{http://python.cz/pyladies/s002-hello-world/and-or.html}

\item Doplň tuhle tabulku:

    {
        \newcommand\rowend{\rule{0pt}{0.5cm}\\ \hline}
        \begin{tabular}{c|c||c|c|c}
        \arrayrulecolor{silver}
        \verb+a+ & \verb+b+ & \verb+a and b+ & \verb+a or b+ & \verb+not a+ \\
        \arrayrulecolor{black}\hline\arrayrulecolor{silver}
        \True & \True &  &  &  \rowend
        \False & \True &  &  &  \rowend
        \True & \False &  &  &  \rowend
        \False & \False &  &  &  \rowend
        \end{tabular}
    }

    \emph{\small Tohle je takzvaná \emph{pravdivostní tabulka}.
          Obsahuje jeden řádek pro každou kombinaci booleovských hodnot, které \texttt{a} a \texttt{b}
          můžou mít. Když se stane, že se v \texttt{and} a \texttt{or} ztratíš,
          doporučuji napsat si podobnou tabulku, a na každý řádek se podívat zvlášť.}

\item Zkus přepsat Kámen, Nůžky, Papír pomocí \texttt{and} a \texttt{or}.
    \\Dokážeš docílit toho, aby se každý z řetězců \texttt{'Plichta.'},
    \texttt{'Počítač vyhrál.'} a \texttt{'Vyhrála jsi!'} objevil v programu jen jednou,
    aniž bys tyhle řetězce musela přiřazovat do proměnných?
    \\Pokud ano, gratuluji!

\item Zkus program „šťastná/bohatá“ přepsat pomocí zanořených \texttt{if}ů. Která verze ti připadá čitelnější?

\end{enumerate}

\end{document}
