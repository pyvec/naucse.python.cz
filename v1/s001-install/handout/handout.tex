% Typeset using lualatex

\documentclass[a4paper,10pt]{article}
\usepackage[utf8]{luainputenc}
\usepackage[pdfborder={0 0 0}]{hyperref}

\usepackage[czech]{babel}

\usepackage[nodayofweek]{datetime}

\usepackage{fouriernc}
\usepackage{tgpagella}
\usepackage{xcolor}
\usepackage[activate=true]{microtype}
\usepackage{amssymb}
\usepackage{fontspec}

\usepackage{calc}
\usepackage{array}
\usepackage{enumitem}
\usepackage{tikz}
\usepackage[a4paper,margin=2cm]{geometry}

\setmainfont[Numbers=OldStyle]{TeX Gyre Pagella}

\definecolor{plhome}{HTML}{CCEEFF}

\pagestyle{empty}

\parskip=1ex
\parindent=0pt

\setlist[enumerate,1]{start=0}

\newcommand\plpage{
    \newpage
    \begin{tikzpicture}[remember picture,overlay]
    \node[opacity=0.1,below left,xshift=1cm,yshift=1cm] at (current page.north east) {\includegraphics[width=12cm]{../../images/pylady-pink}};
    \end{tikzpicture}
}

\newcommand\answerspace{\\\rule[0cm]{0pt}{1cm}}
\newcommand\answerline{\_\_\_\_\_\_\_\_\_\_\_\_\_\_\_\_\_\_\_\_\_\_\_\_\_\_\_\_\_\_\_\_\_\_\_\_\_\_\_\_\_\_}

\begin{document}

\plpage

Vítej v PyLadies!

První sraz bude trochu zvláštní: není co opakovat z minula,
a dnešní hlavní úkol je nainstalovat Python, což se na každém počítači dělá trochu jinak.

Jméno wi-fi sítě a heslo k ní je napsáno v místnosti.
\\Zadání taky, ale pro úplnost: je na \url{http://pyladies.cz/course.html}

Seznam se se sousedkami a začni s instalací; jakékoli otázky rádi zodpoví pomocníci.
V 18:00, až se sejdeme všechny, povím pár slov na úvod a naučíme se základy práce
s příkazovou řádkou. Pak instalaci dokončíme.

Poznámky si piš na tento papír – horní polovinu vyplň podle zadání; spodní je celá pro tebe!

\bigskip\bigskip\hrule\bigskip\bigskip

Můj adresář se soubory k PyLadies je: \answerline
\\(v materiálech bude označovaný jako \colorbox{plhome}{\textasciitilde/pyladies})

\bigskip

Příkazy k puštění Pythonu na mém počítači jsou:

\begin{tabular}{rl<{\rule{0pt}{1cm}}}
0. Otevření příkazové řádky: &  \answerline \\
1. Puštění virtuálního prostředí: & \answerline \\
2. Puštění Pythonu: & \color{blue}{\texttt{python}}\\
\end{tabular}

\bigskip\bigskip\hrule\bigskip\bigskip

\plpage

\section*{Domácí úkoly}

\begin{enumerate}
\item Doma (nebo pár dní po instalaci, pokud nejsi na srazu) si znovu pusť Python,
    a zkus jestli stále funguje.
    \begin{verbatim}
    >>> 1 + 1
    2
    \end{verbatim}
\item Funguje i odečítání?
    \begin{verbatim}
    >>> 4 - 2
    \end{verbatim}
\item A co násobení? Programátoři nenásobí pomocí · ani ×, použijí jiný \emph{operátor}: hvězdičku.
\item Dělení? Znak ÷ se taky na klávesnici těžko píše (zvlášť na české). Jak se asi bude dělit?
\item Závorky v Pythonu fungují jako v matematice.
    Zkus pomocí Pythonu vypočítat: $3 + (4 + 6) \times 8 \div 2 - 1 = $

    Jak se to zapíše v Pythonu?
    \answerspace
\item Jsou i jiné operátory než $+$, $-$, a ty pro násobení a dělení.
    \\Co dělá s čísly operátor \verb+%+ (procento)?
\item A co dělá operátor \verb+**+ (dvě hvězdičky)?
\item Až budeš příště sahat po kalkulačce, použij místo ní Python.

    \bigskip\bigskip\hrule\bigskip\bigskip

\item Python umí i jiné věci než čísla. Třeba takové \emph{řetězce} – slova, věty, nebo jiný text.
    Řetězce se zadávají v uvozovkách – jednoduchých \verb|'| nebo dvojitých \verb|"|:
    \begin{verbatim}
    >>> 'Ahoj!'
    'Ahoj!'
    \end{verbatim}
\item Řetězce jdou spojovat sčítáním:
    \begin{verbatim}
    >>> 'A' + "B"
    'AB'
    \end{verbatim}
\item Co je tady špatně? Jak to spravit?
    \begin{verbatim}
    >>> 'Ahoj' + 'PyLadies!'
    \end{verbatim}
\item Řetězce se dají sčítat. Dají se i násobit? Dělit? Odečítat?
    \answerspace
\item Co se stane, když se pokusím sečíst číslo a řetězec?
    \answerspace
\item A vynásobit?

\end{enumerate}

\end{document}

