% Typeset using lualatex

\documentclass[a4paper,10pt]{article}
\usepackage[utf8]{luainputenc}
\usepackage[pdfborder={0 0 0}]{hyperref}

\providecommand\tr[2]{#2}
\tr{}{\usepackage[czech]{babel}}

\usepackage[nodayofweek]{datetime}

\usepackage{fouriernc}
\usepackage{tgpagella}
\usepackage{xcolor}
\usepackage[activate=true]{microtype}
\usepackage{amssymb}
\usepackage{fontspec}

\usepackage{calc}
\usepackage{array}
\usepackage{enumitem}
\usepackage{tikz}
\usepackage[a4paper,margin=2cm]{geometry}

\setmainfont[Numbers=OldStyle]{TeX Gyre Pagella}

\definecolor{plhome}{HTML}{CCEEFF}

\pagestyle{empty}

\parskip=1ex
\parindent=0pt

\setlist[enumerate,1]{start=0}

\newcommand\plpage{
    \newpage
    \begin{tikzpicture}[remember picture,overlay]
    \node[opacity=0.1,below left,xshift=1cm,yshift=1cm] at (current page.north east) {\includegraphics[width=12cm]{pylady-blue}};
    \end{tikzpicture}
}

\newcommand\answerspace{\\\rule[0cm]{0pt}{1cm}}

\begin{document}

\plpage

Vítej, pomocnice/pomocníku PyLadies!

\bigskip\bigskip

Co děláme:

\begin{itemize}
\item Radíme, když je potřeba
\item Usmíváme se
\item Používáme češtinu, ne žargon
\item Když něco vysvětlíme, zkontrolujeme, že tomu účastnice rozuměla; jestli ne, vysvětlíme to jinak
\item Podporujeme experimentování (i když nevede k ničemu co je v „osnovách“)
\item Podporujeme otázky
\item Podporujeme spolupráci
\item Pokud něco nevíme, přiznáme to
\item Mluvíme pomalu
\item Čekáme na otázky a komentáře (počítej do deseti)
\item Pokud to jde špatně, je to vina je mentorů a materiálů
    \\\emph{(mimochodem: mentoři se taky učí, a materiály jsou na GitHubu)}
\end{itemize}

Předpokládej, že každá účastnice má nulové znalosti ale nekonečnou inteligenci.

\bigskip\bigskip

Co říkáme:

\begin{itemize}
\item Chybovat je lidské
\item Když tě počítač začne frustrovat, udělej si přestávku
\item „Jak to jde?“ „Všechno v pořádku?“
\item „Jistě že to zvládneš!“
\item Všechno, co PyLadies udělají, je skvělé a krásné
\end{itemize}

Příště to bude ještě skvělejší a krásnější.

\bigskip\bigskip

Co neděláme:

\begin{itemize}
\item Nikoho nebalíme a nemáme sexistické narážky (ani pokud se zdají být vtipné)
\item Nedobracíme oči v sloup a nesmějeme se otázkám (otázky \emph{nikdy} nejsou hloupé)
\item Nepropagujeme svoje zaměstnavatele, práci, nebo sebe samotné (pokud to není v rámci poděkování sponzorům)
\item Ničemu se nevysmíváme (ani PHP)
\item Nedotýkáme se klávesnic
\end{itemize}

Klávesnice účastnic jsou \emph{\color{red}{z lávy}}. Nesahej na ně!

\vfill

\hfill \small opsáno od Open Tech School - \url{http://opentechschool.github.io/slides/presentations/coaching}


\end{document}


