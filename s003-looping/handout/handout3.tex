% Typeset using lualatex

\documentclass[a4paper,10pt]{article}
\usepackage[utf8]{luainputenc}
\usepackage[pdfborder={0 0 0}]{hyperref}

\usepackage[czech]{babel}

\usepackage{fouriernc}
\usepackage{tgpagella}
\usepackage{xcolor}
\usepackage[activate=true]{microtype}
\usepackage{amssymb}
\usepackage{fontspec}

\usepackage{calc}
\usepackage{multicol}
\usepackage{array}
\usepackage{enumitem}
\usepackage{tikz}
\usepackage{siunitx}
\usepackage{colortbl}
\usepackage[a4paper,margin=2cm,top=1cm]{geometry}

\setmainfont[Numbers=OldStyle]{TeX Gyre Pagella}

\definecolor{plhome}{HTML}{CCEEFF}
\definecolor{silver}{HTML}{DDDDDD}

\sisetup{
    output-decimal-marker={,}
}
\pagestyle{empty}

\parskip=1ex
\parindent=0pt

\setlist[enumerate,1]{start=0}

\newcommand\plpage{
    \newpage
    \begin{tikzpicture}[remember picture,overlay]
    \node[opacity=0.1,below left,xshift=1cm,yshift=1cm] at (current page.north east) {\includegraphics[width=12cm]{../../images/pylady-pink}};
    \end{tikzpicture}
}

\newcommand\answerspace{\\\rule[0cm]{0pt}{1cm}}

\newcommand\True{\texttt{True}}
\newcommand\False{\texttt{False}}

\begin{document}

\plpage

\section*{Třetí sada domácích úkolů}

\begin{enumerate}
\item Co dělá funkce \verb+print+?

\item Co \emph{vrací} funkce \verb+print+?

\item Co dělá pojemnovaný argument \texttt{end} funkce \verb+print+?

\item Co dělá pojemnovaný argument \texttt{sep} funkce \verb+print+?

\item Čím se liší chyby, které dostaneš když zadáš tyhle příkazy?
    \\\verb+int('blabla')+
    \\\verb+float('blabla')+
    \\\verb+int('8.9')+
    \\\verb+int(8.9)+

\item Jaké znáš typy proměnných?
    \answerspace

\item Jakzaokrouhlí funkce \verb+round+ tato čísla?
    \\\verb+3+
    \\\verb+3,0+
    \\\verb+3,141+
    \\\verb+2,718+
    \\\verb+-8,3+
    \\\verb+3,5+
    \\\verb+4,5+

\item Doplň do hry Kámen, Nůžky, Papír náhodné vybírání tahu počítače.

\end{enumerate}


\begin{enumerate}[resume]
\item Kolik úkolů je v téhle (třetí) sadě?

\end{enumerate}


\end{document}

