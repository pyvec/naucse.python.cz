% Typeset using lualatex

\documentclass[a4paper,10pt]{article}
\usepackage[utf8]{luainputenc}
\usepackage[pdfborder={0 0 0}]{hyperref}

\usepackage[czech]{babel}

\usepackage{fouriernc}
\usepackage{tgpagella}
\usepackage{xcolor}
\usepackage[activate=true]{microtype}
\usepackage{amssymb}
\usepackage{fontspec}

\usepackage{soul}
\usepackage{calc}
\usepackage{multicol}
\usepackage{array}
\usepackage{enumitem}
\usepackage{tikz}
\usepackage{siunitx}
\usepackage{colortbl}
\usepackage{everypage}
\usepackage[a4paper,margin=2cm,top=1cm]{geometry}

\setmainfont[Numbers=OldStyle]{TeX Gyre Pagella}

\definecolor{plpink}{HTML}{FF3232}
\definecolor{plhome}{HTML}{CCEEFF}
\definecolor{silver}{HTML}{888888}

\sisetup{
    output-decimal-marker={,}
}
\pagestyle{empty}

\parskip=1ex
\parindent=0pt

\setlist[enumerate,1]{start=0}

\AddEverypageHook{
    \begin{tikzpicture}[remember picture,overlay]
        \node[opacity=0.1,below left,xshift=1cm,yshift=1cm] at (current page.north east) {\includegraphics[width=12cm]{../../images/pylady-pink}};
        \node[below left,align=right,yshift=-1.5em] at (current page.north east) {\color{white} strana \thepage};
        \node[below left,align=right] at (current page.north east) {\color{white} sada \plsetno};
    \end{tikzpicture}
}

\newcommand\answerspace{\\\rule[0cm]{0pt}{1cm}}

\newcommand\True{\texttt{True}}
\newcommand\False{\texttt{False}}

\usetikzlibrary{turtle}
\newcommand\turtle[1]{\tikz[>=stealth,x=0.25mm,y=0.25mm]\draw[turtle={home,rt=90,#1}];}
\newcommand\gulp[1]{}

\newcommand\plsetno{6}

\begin{document}

\section*{Domácí úkoly \plsetno}

Domácí zvířata jsou např.: \verb+"pes", "kočka", "králík", "had"+.

\begin{enumerate}

\item Napiš funkci, která vypíše jména domácích zvířat, která jsou kratší než 5 písmen.

\item Napiš funkci, která vypíše jména domácích zvířat, která začínají na k.

\item Napiš funkci, která dostane slovo a zjistí,
    jestli je v seznamu domácích zvířat.

\item Napiš funkci, který převede římské říslice na číslo (\verb+int+).
    \\Napiš (a pusť) k této funkci testy, aby sis ověřila že funguje správně.

\item Napiš funkci, která dostane dva seznamy jmen zvířat, a vrátí tři seznamy:
    \begin{enumerate}
        \item Zvířata, která jsou v obou seznamech
        \item Zvířata, která jsou v jen prvním seznamu
        \item Zvířata, která jsou v jen druhém seznamu
    \end{enumerate}
    Napiš (a pusť) k této funkci testy, aby sis ověřila že funguje správně.

\end{enumerate}

\hrule
Dadaistický koutek.
Vyber si básničku, která má aspoň tři sloky po aspoň třech verších.

\begin{enumerate}[resume]

\item Napiš program, který vypíše básničku ze souboru \verb+basnicka.txt+,
    ale obrátí pořadí veršů (t.j. jako první vypíše poslední řádek, atd.)

\item Napiš program, který obrátí pořadí slov v jednotlivých verších.

\item Obrať pořadí slok (ty by měly být oddělené jedním prázdným řádkem).

\item Vypiš slova básně v náhodném pořadí.
    Snaž se co nejlépe zachovat strukturu básně
    (sloky, verše, interpunkci, velká písmena, ...)

\end{enumerate}

\hrule
Úkoly \ref{snakestart}-\ref{snakeend} závisí jeden na druhém, řeš je postupně.

\begin{enumerate}[resume]

\item \label{snakestart}
    Napiš funkci, která dostane seznam souřadnic (párů čísel menších než 10),
    a vypíše je jako mapu. Například:

\verb+nakresli_mapu([(0, 0), (1, 0), (2, 2), (4, 3), (8, 9)])+
\begin{verbatim}
. . . . . . . . X .
. . . . . . . . . .
. . . . . . . . . .
. . . . . . . . . .
. . . . . . . . . .
. . . . . . . . . .
. . . . X . . . . .
. . X . . . . . . .
. . . . . . . . . .
X X . . . . . . . .
\end{verbatim}

\item Napiš funkci \verb+pohyb+, která dostane seznam souřadnic a světovou stranu
    (\verb+"s"+, \verb+"j"+, \verb+"v"+ nebo \verb+"z"+),
    a přidá k seznamu poslední bod „posunutý“ v daném směru. Např.:

\begin{verbatim}
pohyb([(0, 0)], 's') == [(0, 0), (0, 1)]
pohyb([(0, 0), (0, 1)], 's') == [(0, 0), (0, 1), (0, 2)]
pohyb([(0, 0), (0, 1), (0, 2)], 'z') == [(0, 0), (0, 1), (0, 2), (1, 2)]
pohyb([(0, 0), (0, 1), (0, 2), (1, 2)], 'j') == [(0, 0), (0, 1), (0, 2), (1, 2), (1, 1)]
\end{verbatim}
    Funkce by neměla nic vracet.
    Nezapomeň na testy.

\item Napiš cyklus, který se bude ptát uživatele na světovou stranu,
    a podle ní zavolá \verb+pohyb+, vykreslí seznam jako mapu,
    a opět se zeptá na stranu.
    \\Začínej se seznamem \verb+[(0, 0), (0, 1), (0, 2)]+

\item \label{snaketail}
    Doplň funkci \verb+pohyb+ tak, aby při pohybu umazala první bod ze seznamu
    souřadnic. Výsledný seznam tak bude mít stejnou délku, jako před voláním.
    \\Uprav testy.

\item Doplň funkci \verb+pohyb+ tak, aby zamezila
    \begin{itemize}
    \item pohybu ven z mapy
    \item pohybu na políčko, které už v seznamu je
    \end{itemize}
    Vhodná výjimka pro tyto situace je \verb+ValueError('Game over')+.
    \\Doplň i testy.

\item Seznam ovoce obsahuje na začátku jedno ovoce, na políčku na kterém není had.
    Když had sežere ovoce, vyroste („nesmaže“ se mu ocas, viz úkol \ref{snaketail}),
    a pokud na mapě zrovna není další ovoce, na náhodném místě (kde není had) vyroste ovoce nové.
    \\Každých 30 tahů vyroste nové ovoce samo od sebe.
    \\Na mapě se toto tajemné ovoce zobrazuje jako otazník (\verb+?+).

\item \label{snakeend}
    Hadí hřiště musí mít libovolné rozměry větší než 1×4.
    Třeba 20×20 nebo 10×30.

\end{enumerate}

\hrule

\begin{enumerate}[resume]

\item \label{kanacountstart}
    Můj bratr, který píše japonské texty, mě požádal o program
    který počítá znaky.
    Program přečte soubor \verb+rozsypanycaj.txt+, a zjistí,
    kolik znaků je psáno kterou z japonských slabikových abeced:
    \begin{itemize}
    \item {\fontspec{WenQuanYi Zen Hei} ひらがな \emph{(hiragana)}:
        \\ ぁ あ ぃ い ぅ う ぇ え ぉ お か が き ぎ く
        ぐ け げ こ ご さ ざ し じ す ず せ ぜ そ ぞ た
        だ ち ぢ っ つ づ て で と ど な に ぬ ね の は
        ば ぱ ひ び ぴ ふ ぶ ぷ へ べ ぺ ほ ぼ ぽ ま み
        む め も ゃ や ゅ ゆ ょ よ ら り る れ ろ ゎ わ
        ゐ ゑ を ん ゔ ゕ ゖ}
    \item {\fontspec{WenQuanYi Zen Hei} カタカナ \emph{(katakana)}:
        \\ ァ ア ィ イ ゥ ウ ェ エ ォ オ カ ガ キ ギ ク
        グ ケ ゲ コ ゴ サ ザ シ ジ ス ズ セ ゼ ソ ゾ タ
        ダ チ ヂ ッ ツ ヅ テ デ ト ド ナ ニ ヌ ネ ノ ハ
        バ パ ヒ ビ ピ フ ブ プ ヘ ベ ペ ホ ボ ポ マ ミ
        ム メ モ ャ ヤ ュ ユ ョ ヨ ラ リ ル レ ロ ヮ ワ
        ヰ ヱ ヲ ン ヴ ヵ ヶ ヷ ヸ ヹ ヺ}
    \end{itemize}
    (Abecedy budou ke zkopírování na stránkách s materiály.)

    Program vypíše dvě čísla – počet znaků pro každou z abeced.
    Znaky, které nejsou v jedné z abeced, ignoruj.

    Pro testování si jako \verb+rozsypanycaj.txt+ můžeš uložit stránku
    \href{http://ja.wikipedia.org/wiki/%E6%97%A5%E6%9C%AC%E8%AA%9E%E3%81%AE%E8%A1%A8%E8%A8%98%E4%BD%93%E7%B3%BB}{
        http://ja.wikipedia.org/wiki/{\fontspec{WenQuanYi Zen Hei}日本語の表記体系}}
    (odkaz bude také na stránkách).

    Program pošli na soukromý e-mail organizátora (ne do diskusní skupiny).
    Pošli ho jako přílohu, nekopíruj ho do textu e-mailu.
    \emph{(Já ho pak přepošlu bráchovi.)}

\end{enumerate}

\end{document}

