% Typeset using lualatex

\documentclass[a4paper,10pt]{article}
\usepackage[utf8]{luainputenc}
\usepackage[pdfborder={0 0 0}]{hyperref}

\usepackage[czech]{babel}

\usepackage{fouriernc}
\usepackage{tgpagella}
\usepackage{xcolor}
\usepackage[activate=true]{microtype}
\usepackage{amssymb}
\usepackage{fontspec}
\usepackage{titling}
\usepackage{titlesec}

\usepackage{calc}
\usepackage{multicol}
\usepackage{array}
\usepackage{enumitem}
\usepackage{tikz}
\usepackage{siunitx}
\usepackage{colortbl}
\usepackage{everypage}
\usepackage[a4paper,margin=2cm,top=1cm]{geometry}

\setmainfont[Numbers=OldStyle]{TeX Gyre Pagella}

\definecolor{plpink}{HTML}{FF3232}
\definecolor{plhome}{HTML}{CCEEFF}
\definecolor{silver}{HTML}{888888}

\sisetup{
    output-decimal-marker={,}
}
\pagestyle{empty}

\parskip=1ex
\parindent=0pt

\setlist[enumerate,1]{start=0}

\AddEverypageHook{
    \begin{tikzpicture}[remember picture,overlay]
        \node[opacity=0.1,below left,xshift=1cm,yshift=1cm] at (current page.north east) {\includegraphics[width=12cm]{../../images/pylady-pink}};
        \node[below left,align=right,yshift=-1.5em] at (current page.north east) {\color{white} strana \thepage};
        \node[below left,align=right] at (current page.north east) {\color{white} sada \plsetno};
    \end{tikzpicture}
}

\newcommand\answerspace{\\\rule[0cm]{0pt}{1cm}}

\newcommand\True{\texttt{True}}
\newcommand\False{\texttt{False}}

\usetikzlibrary{turtle}
\newcommand\turtle[1]{\tikz[>=stealth,x=0.25mm,y=0.25mm]\draw[turtle={home,rt=90,#1}];}
\newcommand\gulp[1]{}

\newcommand\plsetno{7}

\newcommand\startsection[1]{
     \vspace{0.2ex}
    \hrule
    {\fontspec{Oxygen} \tiny
     \vspace{-1ex}
     \emph{#1}
     \vspace{-1.5em}
    }
}

\newfontfamily\headingfont[]{Bree Serif}
\titleformat*{\section}{\LARGE\headingfont}

\begin{document}

\section*{Domácí projekty \plsetno}

\startsection{Dnešní projekty jsou tak trochu navíc. Udělej je, pokud už máš hotové 1-D piškvorky.}

\begin{enumerate}[resume]

\item
    Napiš program, který zkoupíruje soubor na jiné místo.
    (Zeptá se na půodní jméno a nové jméno, načte původní soubor,
    a zapíše obsah do nového.)
    \\\emph{\small Program zkoušej v adresáři, který neobsahuje nic
        důležitého – pro Python není těžké přepsat soubor,
        ve kterém jsou důležité informace!}

\item
    Můj bratr, který píše japonské texty, mě požádal o program
    který počítá znaky.
    Program přečte soubor \verb+rozsypanycaj.txt+, a zjistí,
    kolik znaků je psáno kterou z japonských slabikových abeced:
    \begin{itemize}
    \item {\fontspec{WenQuanYi Zen Hei} ひらがな \emph{(hiragana)}:
        \\ ぁ あ ぃ い ぅ う ぇ え ぉ お か が き ぎ く
        ぐ け げ こ ご さ ざ し じ す ず せ ぜ そ ぞ た
        だ ち ぢ っ つ づ て で と ど な に ぬ ね の は
        ば ぱ ひ び ぴ ふ ぶ ぷ へ べ ぺ ほ ぼ ぽ ま み
        む め も ゃ や ゅ ゆ ょ よ ら り る れ ろ ゎ わ
        ゐ ゑ を ん ゔ}
    \item {\fontspec{WenQuanYi Zen Hei} カタカナ \emph{(katakana)}:
        \\ ァ ア ィ イ ゥ ウ ェ エ ォ オ カ ガ キ ギ ク
        グ ケ ゲ コ ゴ サ ザ シ ジ ス ズ セ ゼ ソ ゾ タ
        ダ チ ヂ ッ ツ ヅ テ デ ト ド ナ ニ ヌ ネ ノ ハ
        バ パ ヒ ビ ピ フ ブ プ ヘ ベ ペ ホ ボ ポ マ ミ
        ム メ モ ャ ヤ ュ ユ ョ ヨ ラ リ ル レ ロ ヮ ワ
        ヰ ヱ ヲ ン ヴ ヵ ヶ ヷ ヸ ヹ ヺ}
    \end{itemize}
    (Abecedy budou ke zkopírování na stránkách s materiály.)

    Program vypíše dvě čísla – počet znaků pro každou z abeced.
    Znaky, které nejsou v jedné z abeced, ignoruj.

    Pro testování si jako \verb+rozsypanycaj.txt+ můžeš uložit stránku
    \href{http://ja.wikipedia.org/wiki/%E6%97%A5%E6%9C%AC%E8%AA%9E%E3%81%AE%E8%A1%A8%E8%A8%98%E4%BD%93%E7%B3%BB}{
        http://ja.wikipedia.org/wiki/{\fontspec{WenQuanYi Zen Hei}日本語の表記体系}}
    (odkaz bude také na stránkách).

    Podle toho, jak jste se na sraze domluvili, pošli řešení e-mailem
    (např. organizátorům, koučovi, nebo vůbec).
    Posílej ho jako přílohu, nekopíruj ho do textu e-mailu.
    \\Jestli procházíš-li kurz sama a můžeš programování konzultovat s někým
    zkušenějším, je tento úkol na takovou konzultaci ideální téma.

\end{enumerate}

\startsection{Další projekt je rozšíření Šibenice z minulé sady. Jestli ji ještě
    nemáš hotovou, vrať se nejdřív zpět.}

\begin{enumerate}[resume]

\item
    Načítej „obrázky” v Šibenici ze souborů, místo řetězců ve zdrojovém kódu.
    \\\emph{\small Funguje? Nasdílej to na GitHub!}

\end{enumerate}

\end{document}
