% Typeset using lualatex

\documentclass[a4paper,10pt]{article}
\usepackage[utf8]{luainputenc}
\usepackage[pdfborder={0 0 0}]{hyperref}

\usepackage[czech]{babel}

\usepackage{fouriernc}
\usepackage{tgpagella}
\usepackage{xcolor}
\usepackage[activate=true]{microtype}
\usepackage{amssymb}
\usepackage{fontspec}
\usepackage{titling}
\usepackage{titlesec}

\usepackage{calc}
\usepackage{multicol}
\usepackage{array}
\usepackage{enumitem}
\usepackage{tikz}
\usepackage{siunitx}
\usepackage{colortbl}
\usepackage{everypage}
\usepackage[a4paper,margin=2cm,top=1cm]{geometry}

\setmainfont[Numbers=OldStyle]{TeX Gyre Pagella}

\definecolor{plpink}{HTML}{FF3232}
\definecolor{plhome}{HTML}{CCEEFF}
\definecolor{silver}{HTML}{888888}

\sisetup{
    output-decimal-marker={,}
}
\pagestyle{empty}

\parskip=1ex
\parindent=0pt

\setlist[enumerate,1]{start=0}

\AddEverypageHook{
    \begin{tikzpicture}[remember picture,overlay]
        \node[opacity=0.1,below left,xshift=1cm,yshift=1cm] at (current page.north east) {\includegraphics[width=12cm]{../../images/pylady-pink}};
        \node[below left,align=right,yshift=-1.5em] at (current page.north east) {\color{white} strana \thepage};
        \node[below left,align=right] at (current page.north east) {\color{white} sada \plsetno};
    \end{tikzpicture}
}

\newcommand\answerspace{\\\rule[0cm]{0pt}{1cm}}

\newcommand\True{\texttt{True}}
\newcommand\False{\texttt{False}}

\usetikzlibrary{turtle}
\newcommand\turtle[1]{\tikz[>=stealth,x=0.25mm,y=0.25mm]\draw[turtle={home,rt=90,#1}];}
\newcommand\gulp[1]{}

\newcommand\plsetno{5}

\newcommand\startsection[1]{
     \vspace{0.2ex}
    \hrule
    {\fontspec{Oxygen} \tiny
     \vspace{-1ex}
     \emph{#1}
     \vspace{-1.5em}
    }
}

\newfontfamily\headingfont[]{Bree Serif}
\titleformat*{\section}{\LARGE\headingfont}

\begin{document}

\section*{Domácí projekty \plsetno}

Nezapomeň: Když něco funguje, přidej to do Gitu!
\\\emph{\small Pokud materiály procházíš v jiném pořadí než my na srazech, a Git ještě neznáš, můžeš všechny poznámky o Gitu ignorovat.}

\vspace{1cm}

\startsection{Trocha experimentování. Zkus se zamyslet, jestli jsi „dobře” pochopila otázku.}

\begin{enumerate}[resume]

\item Co se stane, když tělo nějaké funkce necháme prázdné?

\item Co se stane, když necháme prázdné tělo cyklu?

\end{enumerate}

\startsection{Procvičení funkcí. Jestli jsi Pythoní funkce nepsala už před kurzem, tak první z těchto projektů určitě udělej. Druhý jen pokud máš ráda geometrii :)}

\begin{enumerate}[resume]

\item Napiš funkci, která vykreslí domeček dané velikosti.
    \\\emph{\small (t.j. velikost se zadá argumentem)}

\gulp{
from math import sqrt
from texturtle import ctx, forward, left, right

def domecek(velikost):
    for i in range(4):
        forward(velikost)
        left(90)
    left(45)
    for n in 1, 2, 2, 1:
        forward(velikost*sqrt(2)/n)
        left(90)
    right(45)

with ctx():
    for n in 10, 20, 50:
        domecek(n)
        forward(n + 10)
}
\turtle{fd=10,lt=90,fd=10,lt=90,fd=10,lt=90,fd=10,lt=90,lt=45,fd=14.142135623730951,lt=90,fd=7.0710678118654755,lt=90,fd=7.0710678118654755,lt=90,fd=14.142135623730951,lt=90,rt=45,fd=20,fd=20,lt=90,fd=20,lt=90,fd=20,lt=90,fd=20,lt=90,lt=45,fd=28.284271247461902,lt=90,fd=14.142135623730951,lt=90,fd=14.142135623730951,lt=90,fd=28.284271247461902,lt=90,rt=45,fd=30,fd=50,lt=90,fd=50,lt=90,fd=50,lt=90,fd=50,lt=90,lt=45,fd=70.71067811865476,lt=90,fd=35.35533905932738,lt=90,fd=35.35533905932738,lt=90,fd=70.71067811865476,lt=90,rt=45,fd=60,}
    \\\emph{\small Funguje? Do Gitu s tím!}

\item Máš-li ráda geometrii*, můžeš zkusit dávat domečkové funkci dva argumenty:
    šířku a výšku.
    \\\emph{\small
        Je potřeba si vzpomenout na Pythagorovu větu a funkci tangens.
        Pozor, funkce \texttt{tan} vrací výsletek v radiánech;
        je potřeba ho převést na stupně.
    }
    \\\emph{\small
        * t.j. jestli nemáš ráda geometrii, tak tenhle projekt přeskoč
    }

\gulp{
from math import sqrt, degrees, atan
from texturtle import ctx, forward, left, right

def domecek(sirka, vyska):
    for i in range(2):
        forward(sirka)
        left(90)
        forward(vyska)
        left(90)

    uhel = degrees(atan(vyska / sirka))
    uhlopricka = sqrt(sirka**2 + vyska**2)
    left(uhel)
    forward(uhlopricka)
    left(180 - 2 * uhel)
    forward(uhlopricka / 2)
    left(2 * uhel)
    forward(uhlopricka / 2)
    left(180 - 2 * uhel)
    forward(uhlopricka)
    left(uhel)

with ctx():
    for a, b in (20, 20), (10, 50), (60, 20):
        domecek(a, b)
        forward(a + 10)
}
\turtle{fd=20,lt=90,fd=20,lt=90,fd=20,lt=90,fd=20,lt=90,lt=45.0,fd=28.284271247461902,lt=90.0,fd=14.142135623730951,lt=90.0,fd=14.142135623730951,lt=90.0,fd=28.284271247461902,lt=45.0,fd=30,fd=10,lt=90,fd=50,lt=90,fd=10,lt=90,fd=50,lt=90,lt=78.69006752597979,fd=50.99019513592785,lt=22.61986494804043,fd=25.495097567963924,lt=157.38013505195957,fd=25.495097567963924,lt=22.61986494804043,fd=50.99019513592785,lt=78.69006752597979,fd=20,fd=60,lt=90,fd=20,lt=90,fd=60,lt=90,fd=20,lt=90,lt=18.43494882292201,fd=63.245553203367585,lt=143.13010235415598,fd=31.622776601683793,lt=36.86989764584402,fd=31.622776601683793,lt=143.13010235415598,fd=63.245553203367585,lt=18.43494882292201,fd=70,}

\end{enumerate}

\startsection{Procvičení programování. Často je jednoduché něco napsat, ale dotažení do konce může být časově náročné. Nemáš-li čas, zkus se aspoň zamyslet jak bys projekt vyřešila.}

\begin{enumerate}[resume]

\item Změň program Kámen, Nůžky, Papír tak, aby opakoval hru dokud
    uživatel nezadá \texttt{"konec"}.

\item Změň funkci \texttt{ano\_nebo\_ne} tak,
    aby se místo \texttt{"ano"} se dalo použít i \texttt{"a"},
    místo \texttt{"ne"} i \texttt{"n"},
    a aby se nebral ohled na velikost písmen a mezery před/za odpovědí.
    \\\emph{\small
        Textům jako \texttt{"možná"} nebo \texttt{"no tak určitě"}
        by počítač dál neměl rozumět.
    }

\end{enumerate}

\startsection{Sada zajímavých (snad) prográmků, které bys teď měla být schopná napsat. Nemáš-li čas, zatím je přeskoč.}

\begin{enumerate}[resume]

\item Napiš program, který se zeptá na příjmení uživatelky/uživatele,
    a zkusí podle něj uhodnout její/jeho pohlaví.
    \\\emph{\small Připomínám: Až to bude fungovat, dej to do Gitu!}

\item Najdi na internetu text své oblíbené písně, zkopíruj si ho do řetězce,
    a zjisti, kolikrát je v něm použito písmeno K.
    \\\emph{\small Připomínám: Až to bude fungovat, dej to do Gitu!}

\item Napiš program, který simuluje tuto hru:
    \\První hráč hází kostkou
    \emph{(t.j. vybírají se náhodná čísla od 1 do 6)},
    dokud nepadne šestka.
    Potom hází další hráč, dokud nepadne šestka i jemu.
    Potom hází hráč třetí, a nakonec čtvrtý.
    Vyhrává ten, kdo na hození šestky potřeboval nejvíc hodů.
    (V případě shody vyhraje ten, kdo házel dřív.)
    \\Program by měl vypisovat všechny hody, a nakonec napsat, kdo vyhrál.
    \\\emph{\small Připomínám: Až to bude fungovat, dej to do Gitu!}

\end{enumerate}

\vfill

Pokračování na další straně. \hfill \emph{\%}

\newpage

\startsection{Nakonec trošku delší projekt. Budeme na něm stavět dál; nedokončíš-li ho teď, budeš ho muset dodělat před příští sadou projektů.}

\begin{center}
\fbox{
\begin{minipage}{8cm}
1-D piškvorky se hrají na řádku s dvaceti políčky.
    \\Hráči střídavě přidávají kolečka (\texttt{o}) a křížky (\texttt{x}),
        třeba:
    \\1. kolo: \texttt{-------\textcolor{plpink}{x}------------}
    \\2. kolo: \texttt{-------x--\textcolor{plpink}{o}---------}
    \\3. kolo: \texttt{-------x\textcolor{plpink}{x}-o---------}
    \\4. kolo: \texttt{-------xx\textcolor{plpink}{o}o---------}
    \\5. kolo: \texttt{------\textcolor{plpink}{x}xxoo---------}
    \\Hráč, která dá tři své symboly vedle sebe, vyhrál.
\end{minipage}
}
\end{center}

\begin{enumerate}[resume]

\item Napiš funkci \texttt{vyhodnot}, která dostane řetězec
    s herním polem 1-D piškvorek,
    a vrátí jednoznakový řetězec podle stavu hry:
    \begin{enumerate}
    \item[\texttt{"x"}] – Vyhrál hráč s křížky (pole obsahuje \texttt{xxx})
    \item[\texttt{"o"}] – Vyhrál hráč s kolečky (pole obsahuje \texttt{ooo})
    \item[\texttt{"!"}] – Remíza (pole neobsahuje \texttt{-}, a nikdo nevyhrál)
    \item[\texttt{"-"}] – Ani jedna ze situací výše
    \end{enumerate}
    \emph{\small Připomínám: Až to bude fungovat, dej to do Gitu!}

\item Napiš funkci \texttt{tah}, která dostane řetězec s herním polem,
    číslo políčka (0-19), a symbol (\texttt{x} nebo \texttt{o}),
    a vrátí herní pole \emph{(t.j. řetězec)} s daným symbolem umístěným na danou pozici.

    Hlavička funkce by tedy měla vypadat nějak takhle:
\\\verb+    def tah(pole, cislo_policka, symbol):+
\\\verb+        "Vrátí herní pole s daným symbolem umístěným na danou pozici"+
\\\verb+        ...+
    \\\emph{Můžeš využít nějakou funkci, kterou jsme napsaly už na sraze?}

\item Napiš funkci \texttt{tah\_hrace}, která dostane řetězec s herním polem,
    zeptá se hráče, na kterou pozici chce hrát, a vrátí herní pole
    se zaznamenaným tahem hráče.
    \\Funkce by měla odmítnout záporná nebo příliš velká čísla,
    a tahy na obsazená políčka.
    Pokud uživatel zadá špatný vstup, funkce mu vynadá a zeptá se znova.
    \\\emph{\small Funguje? Do Gitu s tím!}

\item Napiš funkci \texttt{tah\_pocitace}, která dostane řetězec s herním polem,
    vybere pozici, na kterou hrát, a vrátí herní pole
    se zaznamenaným tahem počítače.
    \\Použij jednoduchou náhodnou „strategii”:
    \begin{enumerate}
    \item[1.] Vyber číslo od 0 do 19
    \item[2.] Pokud je dané políčko volné, hrej na něj
    \item[3.] Pokud ne, opakuj od bodu 1
    \end{enumerate}

    Hlavička funkce by tedy měla vypadat nějak takhle:
\\\verb+    def tah_pocitace(pole):+
\\\verb+        "Vrátí herní pole se zaznamenaným tahem počítače"+
\\\verb+        ...+

\item Napiš funkci \texttt{piskvorky1d}, která vytvoří řetězec s herním polem,
    a střídavě volá funkce \texttt{tah\_hrace} a \texttt{tah\_pocitace},
    dokud někdo nevyhraje nebo nedojde k remíze.
    \\Nezapomeň kontrolovat stav hry po každém tahu.
    \\\emph{\small Funguje? Do Gitu s tím!}

\end{enumerate}

\startsection{Poslední projekt je nepovinný, ale, jak to u podobných projektů bývá, můžeš na něj dostat zpětnou vazbu. Doporučuju toho využít!}

\begin{enumerate}[resume]

\item Zvládneš pro počítač naprogramovat lepší strategii?
    Třeba aby se snažil hrát vedle svých existujících symbolů,
    nebo aby bránil protihráčovi?
    \\Stačí jen docela malé vylepšení!

    Podle toho, jak jste se na sraze domluvili, pošli řešení e-mailem
    (např. organizátorům, koučovi, nebo vůbec).
    Posílej ho jako přílohu, nekopíruj ho do textu e-mailu.
    \\Jestli procházíš-li kurz sama a můžeš programování konzultovat s někým
    zkušenějším, je tento úkol na takovou konzultaci ideální téma.

    \emph{\small A do Gitu to samozřejmě dej taky...}

\end{enumerate}

\end{document}
