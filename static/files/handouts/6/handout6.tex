% Typeset using lualatex

\documentclass[a4paper,10pt]{article}
\usepackage[utf8]{luainputenc}
\usepackage[pdfborder={0 0 0}]{hyperref}

\usepackage[czech]{babel}

\usepackage{fouriernc}
\usepackage{tgpagella}
\usepackage{xcolor}
\usepackage[activate=true]{microtype}
\usepackage{amssymb}
\usepackage{fontspec}
\usepackage{titling}
\usepackage{titlesec}

\usepackage{calc}
\usepackage{multicol}
\usepackage{array}
\usepackage{enumitem}
\usepackage{tikz}
\usepackage{siunitx}
\usepackage{colortbl}
\usepackage{everypage}
\usepackage[a4paper,margin=2cm,top=1cm]{geometry}

\setmainfont[Numbers=OldStyle]{TeX Gyre Pagella}

\definecolor{plpink}{HTML}{FF3232}
\definecolor{plhome}{HTML}{CCEEFF}
\definecolor{silver}{HTML}{888888}

\sisetup{
    output-decimal-marker={,}
}
\pagestyle{empty}

\parskip=1ex
\parindent=0pt

\setlist[enumerate,1]{start=0}

\AddEverypageHook{
    \begin{tikzpicture}[remember picture,overlay]
        \node[opacity=0.1,below left,xshift=1cm,yshift=1cm] at (current page.north east) {\includegraphics[width=12cm]{../../images/pylady-pink}};
        \node[below left,align=right,yshift=-1.5em] at (current page.north east) {\color{white} strana \thepage};
        \node[below left,align=right] at (current page.north east) {\color{white} sada \plsetno};
    \end{tikzpicture}
}

\newcommand\answerspace{\\\rule[0cm]{0pt}{1cm}}

\newcommand\True{\texttt{True}}
\newcommand\False{\texttt{False}}

\usetikzlibrary{turtle}
\newcommand\turtle[1]{\tikz[>=stealth,x=0.25mm,y=0.25mm]\draw[turtle={home,rt=90,#1}];}
\newcommand\gulp[1]{}

\newcommand\plsetno{8}

\newcommand\startsection[1]{
     \vspace{0.2ex}
    \hrule
    {\fontspec{Oxygen} \tiny
     \vspace{-1ex}
     \emph{#1}
     \vspace{-1.5em}
    }
}

\newfontfamily\headingfont[]{Bree Serif}
\titleformat*{\section}{\LARGE\headingfont}

\begin{document}

\section*{Domácí projekty \plsetno}

\startsection{Na začátek trocha práce se seznamy.
    Některé z projektů – a zvlášť ten poslední – potřebují trochu přemýšlení;
    nebudeš-li vědět jak dál, zeptej se na Internetu nebo na sraze ostatních,
    a řešte společně!}

\begin{enumerate}

\item Udělej si seznam domácích zvířat. Budeš ho potřebovat v dalších úlohách.
    \\Domácí zvířata jsou např.: \verb+"pes", "kočka", "králík", "had"+.


\item Napiš funkci, která vypíše jména domácích zvířat, která jsou kratší než 5 písmen.

\item Napiš funkci, která vypíše jména domácích zvířat, která začínají na k.

\item Napiš funkci, která dostane slovo a zjistí,
    jestli je v seznamu domácích zvířat.

\item Napiš funkci, která dostane dva seznamy jmen zvířat, a vrátí tři seznamy:
    \begin{enumerate}
        \item Zvířata, která jsou v obou seznamech
        \item Zvířata, která jsou v jen prvním seznamu
        \item Zvířata, která jsou v jen druhém seznamu
    \end{enumerate}
    Napiš (a pusť) k této funkci testy, aby sis ověřila že funguje správně.

\item Napiš program, který seřadí seznam domácích zvířat podle abecedy.

\item Had byl pyšný na to, že je v abecedě první.
    Dokud nepřiletěla \verb+"andulka"+.
    \\Abys hada uklidnila, vytvoř funkci, která zvířata seřadí podle abecedy,
    ale bude ignorovat první písmeno (t.j. vrátí
        \texttt{["h{\color{plpink}ad}",
                 "p{\color{plpink}es}",
                 "a{\color{plpink}ndulka}",
                 "k{\color{plpink}očka}",
                 "k{\color{plpink}rálík}"]}).

    \emph{
        Postup:
        \begin{itemize}
            \item Máš seznam \emph{hodnot}, které chceš seřadit podle nějakého
                \emph{klíče}. Klíč se dá z každé hodnoty vypočítat.
            \item Vytvoř seznam dvojic \texttt{[klíč, hodnota]}.
            \item Seřaď tento seznam dvojic – dvojice se řadí nejdřív podle
                prvního prvku, pak druhého, atd.
            \item Nakonec vytvoř ze seznamu dvojic opět jen seznam hodnot.
        \end{itemize}
        Proč má zrovna had takovéhle výsadní postavení, zjistíš později.
    }

\end{enumerate}

\startsection{Jedna klasická programovací úloha, která nejspíš pořádně potrápí
    tvé logické myšlení. Je nepovinná, nemáš-li na ni aspoň pár hodin,
    tak ji přeskoč.}

\begin{enumerate}[resume]

\item Napiš funkci, která převede římské číslice na číslo (\verb+int+).
    \\\emph{Nápověda: Nejdřív napiš k této funkci nějaké testy, aby sis
        ověřila že (a co) (ne)funguje správně.}

\end{enumerate}

\startsection{Dadaistický koutek: procvičení práce se seznamy, řetězci,
    a soubory. Opět: nebudeš-li vědět jak dál, zeptej se!}

\begin{enumerate}[resume]

\item Vyber si básničku, která má aspoň tři sloky po aspoň třech verších.
    Ulož ji do souboru \verb+basnicka.txt+,

\item Napiš program, který vypíše básničku ze souboru \verb+basnicka.txt+,
    ale obrátí pořadí veršů (t.j. jako první vypíše poslední řádek, atd.)
    \\\emph{Nápověda: Každý seznam má metodu \texttt{reverse}, která ho „obrátí”.}

\item Napiš program, který obrátí pořadí slov v jednotlivých verších.

\item Obrať pořadí slok (ty by měly být oddělené jedním prázdným řádkem).

\item Vypiš slova básně v náhodném pořadí.
    \\\emph{Bonusový projekt: Snaž se přitom co nejlépe zachovat strukturu básně
        (sloky, verše, interpunkci, velká písmena, ...)}

\end{enumerate}

\vfill

\hfill {\fontspec{Oxygen} \tiny Nestačí? Další projekty jsou na druhé straně...}
\newpage

\startsection{Projekty \ref{snakestart}-\ref{snakeend} závisí jeden na druhém,
    řeš je postupně. Až to uděláš, můžeš si zahrát hru!
    Tahle sekce není jednoduchá (a poslední dva projekty jsou obzvláště náročné).
    Kdyžtak můžeš zkusit spojit síly s ostatními účastnicemi kurzu :)}

\begin{enumerate}[resume]

\item \label{snakestart}
    Napiš funkci, která dostane seznam souřadnic (párů čísel menších než 10),
    a vypíše je jako mapu. Například:

\verb+nakresli_mapu([(0, 0), (1, 0), (2, 2), (4, 3), (8, 9)])+
\\\verb+X X . . . . . . . .+
\\\verb+. . . . . . . . . .+
\\\verb+. . X . . . . . . .+
\\\verb+. . . . X . . . . .+
\\\verb+. . . . . . . . . .+
\\\verb+. . . . . . . . . .+
\\\verb+. . . . . . . . . .+
\\\verb+. . . . . . . . . .+
\\\verb+. . . . . . . . . .+
\\\verb+. . . . . . . . X .+

    \emph{Jak na to?
        \begin{enumerate}
        \item Udělej tabulku (seznam seznamů) se samými tečkami, něco jako: \\
            \texttt{[['.', '.', '.'], ['.', '.', '.'], ['.', '.', '.']]}.
        \item Na příslušných místech nahraď tečky X-ky.
        \item Tabulku vypiš pomocí dvou cyklů \texttt{for} zanořených do sebe.
        \end{enumerate}
    }

\item Napiš funkci \verb+pohyb+, která dostane seznam souřadnic a světovou stranu
    (\verb+"s"+, \verb+"j"+, \verb+"v"+ nebo \verb+"z"+),
    a přidá k seznamu poslední bod „posunutý“ v daném směru. Např.:

  \verb+souradnice = [(0, 0)]+
\\\verb+pohyb(souradnice, 'v')+
\\\verb+print(souradnice)         # → [(0, 0), (0, 1)]+
\\\verb+pohyb(souradnice, 'v')+
\\\verb+print(souradnice)         # → [(0, 0), (0, 1), (0, 2)]+
\\\verb+pohyb(souradnice, 'j')+
\\\verb+print(souradnice)         # → [(0, 0), (0, 1), (0, 2), (1, 2)]+
\\\verb+pohyb(souradnice, 's')+
\\\verb+print(souradnice)         # → [(0, 0), (0, 1), (0, 2), (1, 2), (1, 1)]+

    Funkce by neměla nic vracet.
    Nezapomeň na testy.

\item Napiš cyklus, který se bude ptát uživatele na světovou stranu,
    podle ní zavolá \verb+pohyb+, a následně vykreslí seznam jako mapu.
    Pak se opět se zeptá na stranu, atd.
    \\Začínej se seznamem \verb+[(0, 0), (1, 0), (2, 0)]+

\item \label{snaketail}
    Doplň funkci \verb+pohyb+ tak, aby při pohybu umazala první bod ze seznamu
    souřadnic. Výsledný seznam tak bude mít stejnou délku, jako před voláním.
    \\Uprav testy.

\item Doplň funkci \verb+pohyb+ tak, aby zamezila
    \begin{itemize}
    \item pohybu ven z mapy
    \item pohybu na políčko, které už v seznamu je
    \end{itemize}
    Vhodná výjimka pro tyto situace je \verb+ValueError('Game over')+.
    \\Doplň i testy.

\item Přidej do hry hadí potravu. Tady jsou pravidla pro vegetariánského hada,
    ale můžeš si je změnit podle chuti:
    \\ Seznam ovoce obsahuje na začátku jedno ovoce, na políčku na kterém není had.
    (například: \texttt{[(2, 3)]} znamená jedno ovoce na pozici (2, 3).)
    Když had sežere ovoce, vyroste („nesmaže“ se mu ocas, tedy neprovede se to,
    cos přidala v projektu \ref{snaketail}),
    a pokud na mapě zrovna není další ovoce, na náhodném místě (kde není had) vyroste ovoce nové.
    \\Každých 30 tahů vyroste nové ovoce samo od sebe.
    \\Na mapě se toto tajemné ovoce zobrazuje jako otazník (\verb+?+).

\item \label{snakeend}
    Hadí hřiště může mít libovolné rozměry větší než 4×1.
    Třeba 20×20 nebo 10×30.

\end{enumerate}

\startsection{A nakonec projekt na přemýšlení.}

\begin{enumerate}[resume]

\item Může seznam obsahovat sám sebe? Zkus co nejjednodušeji udělat takový seznam, aby platilo:
    \\\verb+seznam[5][5][5][5][5][5][5][5][5][5][5][5][5][5][5][5][5][5][0] == 5+.

\end{enumerate}

\end{document}
