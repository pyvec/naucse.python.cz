% Typeset using lualatex

\documentclass[a4paper,10pt]{article}
\usepackage[utf8]{luainputenc}
\usepackage[pdfborder={0 0 0}]{hyperref}

\usepackage[czech]{babel}

\usepackage{fouriernc}
\usepackage{tgpagella}
\usepackage{xcolor}
\usepackage[activate=true]{microtype}
\usepackage{amssymb}
\usepackage{fontspec}
\usepackage{titling}
\usepackage{titlesec}

\usepackage{calc}
\usepackage{multicol}
\usepackage{array}
\usepackage{enumitem}
\usepackage{tikz}
\usepackage{siunitx}
\usepackage{colortbl}
\usepackage{everypage}
\usepackage[a4paper,margin=2cm,top=1cm]{geometry}

\setmainfont[Numbers=OldStyle]{TeX Gyre Pagella}

\definecolor{plpink}{HTML}{FF3232}
\definecolor{plhome}{HTML}{CCEEFF}
\definecolor{silver}{HTML}{888888}

\sisetup{
    output-decimal-marker={,}
}
\pagestyle{empty}

\parskip=1ex
\parindent=0pt

\setlist[enumerate,1]{start=0}

\AddEverypageHook{
    \begin{tikzpicture}[remember picture,overlay]
        \node[opacity=0.1,below left,xshift=1cm,yshift=1cm] at (current page.north east) {\includegraphics[width=12cm]{../../images/pylady-pink}};
        \node[below left,align=right,yshift=-1.5em] at (current page.north east) {\color{white} strana \thepage};
        \node[below left,align=right] at (current page.north east) {\color{white} sada \plsetno};
    \end{tikzpicture}
}

\newcommand\answerspace{\\\rule[0cm]{0pt}{1cm}}

\newcommand\True{\texttt{True}}
\newcommand\False{\texttt{False}}

\usetikzlibrary{turtle}
\newcommand\turtle[1]{\tikz[>=stealth,x=0.25mm,y=0.25mm]\draw[turtle={home,rt=90,#1}];}
\newcommand\gulp[1]{}

\newcommand\plsetno{4}

\newcommand\startsection[1]{
     \vspace{0.2ex}
    \hrule
    {\fontspec{Oxygen} \tiny
     \vspace{-1ex}
     \emph{#1}
     \vspace{-1.5em}
    }
}

\newfontfamily\headingfont[]{Bree Serif}
\titleformat*{\section}{\LARGE\headingfont}

\begin{document}

\section*{Domácí projekty {\plsetno}}

Tohle je čtvrtá sada projektů. Udělej si Gitový repozitář, do kterého si ukládej řešení.
Jakmile ti bude nějaké řešení fungovat, ulož si ho jako revizi v Gitu.
\\(Občas je to připomenuto i u jednotlivých projektů.)

\startsection{Na začátek trocha zdánlivě nudného opakování, ať si trochu osvěžíme programování.
To ale neznamená, že tu nemůžeš najít něco zvláštního a překvapivého!}

\begin{enumerate}
\item Co dělá funkce \verb+print+?

\item Co \emph{vrací} funkce \verb+print+?

\item Co dělá pojemnovaný argument \texttt{end} funkce \verb+print+?

\item Co dělá pojemnovaný argument \texttt{sep} funkce \verb+print+?

\item Čím se liší chyby, které dostaneš když zadáš tyhle příkazy?
    \\\verb+int('blabla')+
    \\\verb+float('blabla')+
    \\\verb+int('8.9')+
    \\\verb+int(8.9)+

\end{enumerate}

\startsection{Následující sada projektů není jednoduchá, ale uděláš-li ji, pochopíš různá použití cyklu for.}

\begin{enumerate}[resume]
\item Pomocí cyklu \verb+for+ a funkce \texttt{range()} napiš program, který vypíše:
\begin{verbatim}
    a
    a
    a
    a
\end{verbatim}
    Až to bude fungovat, dej to do Gitu!

\item Pomocí cyklu \verb+for+ napiš program, který vypíše:
\begin{verbatim}
    Řádek 0
    Řádek 1
    Řádek 3
    Řádek 4
\end{verbatim}
    Funguje? Dej to do Gitu!

\item Jak jsi pojmenovala proměnnou, kterou jsi v minulám příkladu použila?
    Vymysli pro ni název, který nejlépe vystihuje, co proměnná obsahuje.
    \\\emph{\small Je docela důležité proměnnou pojmenovat výstižně, jinak se v následujících projektech můžeš ztratit.
            Možné řešení je uvedeno za zadáním.}
    \label{cisloradku}

\item Pomocí cyklu \verb+for+ napiš program, který vypíše:
\\\verb+    0 na druhou je 0+
\\\verb+    1 na druhou je 1+
\\\verb+    2 na druhou je 4+
\\\verb+    3 na druhou je 9+
\\\verb+    4 na druhou je 16+
    \\\emph{\small Jak pojmenuješ proměnnou cyklu?}

\item Pomocí cyklů \verb+for+, a parametru \verb+end+ pro \verb+print+, napiš program který vypíše:
\\\verb+    X X X X X+
\\\verb+    X X X X X+
\\\verb+    X X X X X+
\\\verb+    X X X X X+
\\\verb+    X X X X X+
    \\\emph{\small Jak pojmenuješ proměnnou cyklu? A tu druhou?}

\item Pomocí cyklů \verb+for+, a parametru \verb+end+ pro \verb+print+, napiš program který vypíše:
\\\verb+    0 0 0 0 0+
\\\verb+    0 1 2 3 4+
\\\verb+    0 2 4 6 8+
\\\verb+    0 3 6 9 12+
\\\verb+    0 4 8 12 16+
    \\\emph{\small Funguje? Dej to do Gitu!}

\item Pomocí cyklů \verb+for+, a parametru \verb+end+ pro \verb+print+, napiš program který vypíše:
\\\verb+    X+
\\\verb+    X X+
\\\verb+    X X X+
\\\verb+    X X X X+
    \\\emph{\small Funguje? Do Gitu s tím!}

\item Pomocí cyklu \verb+for+ a příkazu \verb+if+ napiš program, který vypíše následující text. Každý \texttt{print} musí být uvnitř v cyklu:
\\\verb+    první řádek+
\\\verb+    není první+
\\\verb+    není první+
\\\verb+    není první+

\item Pomocí cyklů \verb+for+ a příkazu \verb+if+ napiš program, který vypíše:
\\\verb+    X X X X X X+
\\\verb+    X         X+
\\\verb+    X         X+
\\\verb+    X         X+
\\\verb+    X X X X X X+

\item Programy s cyklem \verb+for+ uprav tak, aby počet řádků
    (či velikost čtverce/trojúhelníku/tabulky)
    mohl zadat uživatel.
    \\\emph{\small Funguje? Do Gitu s tím!}

\end{enumerate}

\startsection{Následující sada projektů může vyžadovat delší zamyšlení. A to zamyšlení je důležitější než samotná odpověď.}

\begin{enumerate}[resume]
\item Co dělá tenhle kód?
\begin{verbatim}
for c in 'Ahoj světe!':
    print(c)
\end{verbatim}

\item Vymyslíš lepší jméno pro proměnnou \verb+c+ z minulé úlohy?

\item Co dělá tenhle kód?
\begin{verbatim}
for c in 38:
    print(c)
\end{verbatim}

\item Už víš, co dělá \texttt{for} s \texttt{range()}, výčtem hodnot, a řetězcem.
    Dokážeš to zobecnit, popsat \texttt{for} jednodušeji než jak je popsán v materiálech?

\end{enumerate}


\startsection{Teď několik programovacích oříšků pro dlouhé chvíle. Nemáš-li čas, přeskoč je.}

\begin{enumerate}[resume]

\item Napiš program, který se zeptá na 3 čísla,
    a zjistí jestli je jejich součet větší než 10.
    \\\emph{\small Funguje? Do Gitu s tím!}

\item Napiš program, který načte číslo a zjistí, jestli je sudé.
    \emph{\small Sudá čísla jsou beze \emph{zbytku} dělitelná dvěma.}

\item Napiš program, který vypíše čísla od jedné do 100, ale:
    \begin{itemize}
        \item Pokud je číslo dělitelné třemi, napíše místo něj „bum”.
        \item Pokud je číslo dělitelné pěti, napíše místo něj „bác”.
        \item Pokud je číslo dělitelné pěti i třemi zároveň, napíše místo toho „bum-bác”.
    \end{itemize}
    \emph{\small Funguje? Do Gitu s tím!}

\item Máš-li ráda matematiku*, a nebojíš-li se výzvy, načti od uživatele číslo $n$ a:
    \begin{itemize}
        \item Vypočti faktoriál $n!$ (součin všech celých čísel od 1 do $n$)
        \item Zjisti, jestli je $n$ prvočíslo
        \item Vypiš prvních $n$ členů Fibonacciho posloupnosti ($1, 1, 2, 3, 5, 8, 13, 21, \ldots $)
    \end{itemize}
    \emph{\small * t.j. nemáš-li ráda matematiku, nedělej tenhle projekt :)}

\end{enumerate}

\startsection{A nakonec něco na oddech (snad)...}

\begin{enumerate}[resume]
\item Tohle je poslední projekt ze čtvrté sady. Kolik je v této sadě projektů?

\end{enumerate}

\vfill
\hfill~%
\begin{tikz}
\node[rotate=180]{Možné řešení projektu \ref{cisloradku}: \texttt{cislo\_radku} };
\end{tikz}

\end{document}
