% Typeset using lualatex

\documentclass[a4paper,10pt]{article}
\usepackage[utf8]{luainputenc}
\usepackage[pdfborder={0 0 0}]{hyperref}

\usepackage[czech]{babel}

\usepackage{fouriernc}
\usepackage{tgpagella}
\usepackage{xcolor}
\usepackage[activate=true]{microtype}
\usepackage{amssymb}
\usepackage{fontspec}

\usepackage{calc}
\usepackage{multicol}
\usepackage{array}
\usepackage{enumitem}
\usepackage{tikz}
\usepackage{siunitx}
\usepackage{colortbl}
\usepackage{everypage}
\usepackage[a4paper,margin=2cm,top=1cm]{geometry}

\setmainfont[Numbers=OldStyle]{TeX Gyre Pagella}

\definecolor{plpink}{HTML}{FF3232}
\definecolor{plhome}{HTML}{CCEEFF}
\definecolor{silver}{HTML}{888888}

\sisetup{
    output-decimal-marker={,}
}
\pagestyle{empty}

\parskip=1ex
\parindent=0pt

\setlist[enumerate,1]{start=0}

\AddEverypageHook{
    \begin{tikzpicture}[remember picture,overlay]
        \node[opacity=0.1,below left,xshift=1cm,yshift=1cm] at (current page.north east) {\includegraphics[width=12cm]{../../images/pylady-pink}};
        \node[below left,align=right,yshift=-1.5em] at (current page.north east) {\color{white} strana \thepage};
        \node[below left,align=right] at (current page.north east) {\color{white} sada \plsetno};
    \end{tikzpicture}
}

\newcommand\answerspace{\\\rule[0cm]{0pt}{1cm}}

\newcommand\True{\texttt{True}}
\newcommand\False{\texttt{False}}

\usetikzlibrary{turtle}
\newcommand\turtle[1]{\tikz[>=stealth,x=0.25mm,y=0.25mm]\draw[turtle={home,rt=90,#1}];}
\newcommand\gulp[1]{}

\newcommand\plsetno{5}

\begin{document}

\section*{Domácí úkoly \plsetno}

\begin{enumerate}
\item Rozděl 1D Piškvorky na čtyři moduly:
    \begin{itemize}
        \item \texttt{ai.py}, kde bude funkce \texttt{tah\_pocitace}
        \item \texttt{piskvorky.py}, kde budou ostatní funkce
        \item \texttt{hra.py}, kde bude import a volání hlavní funkce z \texttt{piskvorky.py} (a nic jiného)
        \item \texttt{test\_piskvorky.py}, kde budou testy
    \end{itemize}

\item Napiš \emph{aspoň} dva testy na každou funkci z Piškvorek, ke které testy napsat umíš.
    \\\emph{\small Testy pravděpodobně neumíš napsat na funkci \texttt{input}, a všechny funkce které ji (byť nepřímo) volají.
            A taky asi neumíš otestovat účinek funkce \texttt{print}}

\item Zkus program přepsat tak, aby část která není pokrytá automatickými testy byla co nejmenší
    \\\emph{\small Místo volání \texttt{print} můžeš vracet řetězec;
            místo volání \texttt{imput} můžeš vstup brát jako argument.
            Samotný \texttt{print} nebo \texttt{input} tak může být až ve volající funkci.}

\item Doplň funkci \texttt{tah\_pocitace} tak, aby brala jako argument symbol, který má udělat – buď \texttt{'x'} nebo \texttt{'o'}.
    \\Doplň testy.

\item Ověř, že se funkce \texttt{tah\_pocitace} se umí vyrovnat s jinou délkou hracího pole, než 20.

\item Ověř si, že se \texttt{tah\_pocitace} chová rozumně když dostane plné hrací pole, nebo pole s délkou 0.
    \\\emph{\small Rozumné chování v tomto případě znamená vyvolání rozumné výjimky.}

\item Pošli svůj soubor \texttt{ai.py} (s funkcí \texttt{tah\_pocitace})
    na soukromý e-mail organizátora (ne do diskusní skupiny).
    Pošli ho jako přílohu, nekopíruj ho do textu e-mailu.
    \\Dozvíš se, jak si tvá strategie stojí proti ostatním – a proti testům, které napíše organizátor :)
    \\Strategii je možné i vylepšovat, a poslat několik různých verzí, takže se neboj!
    \\\emph{\small Neboj se funkci poslat, i kdyby ještě nebyla úplně dodělaná.}
    \\Ceny budou v kategoriích jako:
    \begin{itemize}
        \item nejlepší strategie
        \item největší vylepšení od první verze do poslední
        \item nejlepší ošetřování chyb
        \item ... a možná další, které budou oznámeny na diskuzní skupině
    \end{itemize}

\end{enumerate}

\end{document}

