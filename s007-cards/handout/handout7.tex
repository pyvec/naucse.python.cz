% Typeset using lualatex

\documentclass[a4paper,10pt]{article}
\usepackage[utf8]{luainputenc}
\usepackage[pdfborder={0 0 0}]{hyperref}

\usepackage[czech]{babel}

\usepackage{fouriernc}
\usepackage{tgpagella}
\usepackage{xcolor}
\usepackage[activate=true]{microtype}
\usepackage{amssymb}
\usepackage{fontspec}

\usepackage{calc}
\usepackage{multicol}
\usepackage{array}
\usepackage{enumitem}
\usepackage{tikz}
\usepackage{siunitx}
\usepackage{colortbl}
\usepackage{everypage}
\usepackage[a4paper,margin=2cm,top=1cm]{geometry}

\setmainfont[Numbers=OldStyle]{TeX Gyre Pagella}

\definecolor{plpink}{HTML}{FF3232}
\definecolor{plhome}{HTML}{CCEEFF}
\definecolor{silver}{HTML}{888888}

\sisetup{
    output-decimal-marker={,}
}
\pagestyle{empty}

\parskip=1ex
\parindent=0pt

\setlist[enumerate,1]{start=0}

\AddEverypageHook{
    \begin{tikzpicture}[remember picture,overlay]
        \node[opacity=0.1,below left,xshift=1cm,yshift=1cm] at (current page.north east) {\includegraphics[width=12cm]{../../images/pylady-pink}};
        \node[below left,align=right,yshift=-1.5em] at (current page.north east) {\color{white} strana \thepage};
        \node[below left,align=right] at (current page.north east) {\color{white} sada \plsetno};
    \end{tikzpicture}
}

\newcommand\answerspace{\\\rule[0cm]{0pt}{1cm}}

\newcommand\True{\texttt{True}}
\newcommand\False{\texttt{False}}

\usetikzlibrary{turtle}
\newcommand\turtle[1]{\tikz[>=stealth,x=0.25mm,y=0.25mm]\draw[turtle={home,rt=90,#1}];}
\newcommand\gulp[1]{}

\newcommand\plsetno{7}

\begin{document}

\section*{Domácí úkoly \plsetno}

Udělej úkoly 11-17 z minulé sady. Na příštím srazu je využiješ.
Až to budeš mít hotové, můžeš zkusit udělat pár věcí navíc do Klondike:

\bigskip\bigskip
\hrule

\begin{enumerate}

\item Pokud se odkryje (v balíčku nebo na konci sloupečku) karta, která jde dát
    na cílovou hromádku, tak se tam dá automaticky.
    (A tím se múže odkrýt další taková karta, která se taky automaticky
    uklidí, a tak dál)
    \\\emph{Doporučuju třeba následující cyklus na konci každého tahu v hra.py:}
    \\\verb+while automaticky_tah(hra):+
    \\\verb+    udelej_tah(automaticky_tah(hra))+

\item Na přesouvání několika karet najednou by nemuselo být potřeba zadávat
    počet karet. Pokud sloupec A obsahuje 8, 7, 6, 5, 4 (správných barev)
    a sloupec B končí sedmičkou, aktuálně musíš zadat tah \verb+A3B+.
    Dalo by se to ale zařídit tak, aby po zadání \verb+A3B+ počítač
    sám zjistil, že na B patří šestka, a přesunul tři karty.
    \\\emph{Funkce \texttt{nacti\_tah} může v tomto případě vracet \texttt{pocet=0};
            funkce \texttt{priprav\_tah} pak doplní příslušné číslo.}
    \\Dej si pozor, aby chybová hláška neřekla hráči nic o kartách, které jsou
    rubem nahoru.
    \\A taky, jestli tohle uděláš, nezapomeň upravit nápovědu!

\item A kdyby ses hodně nudila, napiš funkci, která dostane jako argument hru,
    a vrátí tah k zahrání (tj. to co by vrátila funkce \verb+nacti_tah+).
    Pak můžeš nechat počítač hrát automaticky, nebo umožnit hráči
    aby nechal některé tahy na počítači.

\end{enumerate}

\hrule

\begin{enumerate}[resume]

\item Může \emph{n}-tice obsahovat sama sebe? Zkus co nejjednodušeji udělat takovou \emph{n}-tici, aby platilo:
    \\\verb+ntice[5][5][5][5][5][5][5][5][5][5][5][5][5][5][5][5][5][5][0] == 5+.

\end{enumerate}

\end{document}

